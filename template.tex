\documentclass[journal]{vgtc}                % final (journal style)
%\documentclass[review,journal]{vgtc}         % review (journal style)
%\documentclass[widereview]{vgtc}             % wide-spaced review
%\documentclass[preprint,journal]{vgtc}       % preprint (journal style)

%% Uncomment one of the lines above depending on where your paper is
%% in the conference process. ``review'' and ``widereview'' are for review
%% submission, ``preprint'' is for pre-publication, and the final version
%% doesn't use a specific qualifier.

%% Please use one of the ``review'' options in combination with the
%% assigned online id (see below) ONLY if your paper uses a double blind
%% review process. Some conferences, like IEEE Vis and InfoVis, have NOT
%% in the past.

%% Please note that the use of figures other than the optional teaser is not permitted on the first page
%% of the journal version.  Figures should begin on the second page and be
%% in CMYK or Grey scale format, otherwise, colour shifting may occur
%% during the printing process.  Papers submitted with figures other than the optional teaser on the
%% first page will be refused. Also, the teaser figure should only have the
%% width of the abstract as the template enforces it.

%% These few lines make a distinction between latex and pdflatex calls and they
%% bring in essential packages for graphics and font handling.
%% Note that due to the \DeclareGraphicsExtensions{} call it is no longer necessary
%% to provide the the path and extension of a graphics file:
%% \includegraphics{diamondrule} is completely sufficient.
%%
\ifpdf%                                % if we use pdflatex
  \pdfoutput=1\relax                   % create PDFs from pdfLaTeX
  \pdfcompresslevel=9                  % PDF Compression
  \pdfoptionpdfminorversion=7          % create PDF 1.7
  \ExecuteOptions{pdftex}
  \usepackage{graphicx}                % allow us to embed graphics files
  \DeclareGraphicsExtensions{.pdf,.png,.jpg,.jpeg} % for pdflatex we expect .pdf, .png, or .jpg files
\else%                                 % else we use pure latex
  \ExecuteOptions{dvips}
  \usepackage{graphicx}                % allow us to embed graphics files
  \DeclareGraphicsExtensions{.eps}     % for pure latex we expect eps files
\fi%

%% it is recomended to use ``\autoref{sec:bla}'' instead of ``Fig.~\ref{sec:bla}''
\graphicspath{{figures/}{pictures/}{images/}{./}} % where to search for the images

\usepackage{microtype}                 % use micro-typography (slightly more compact, better to read)
\PassOptionsToPackage{warn}{textcomp}  % to address font issues with \textrightarrow
\usepackage{textcomp}                  % use better special symbols
\usepackage{mathptmx}                  % use matching math font
\usepackage{times}                     % we use Times as the main font
\renewcommand*\ttdefault{txtt}         % a nicer typewriter font
\usepackage{cite}                      % needed to automatically sort the references
\usepackage{tabu}                      % only used for the table example
\usepackage{booktabs}                  % only used for the table example
%% We encourage the use of mathptmx for consistent usage of times font
%% throughout the proceedings. However, if you encounter conflicts
%% with other math-related packages, you may want to disable it.

%% In preprint mode you may define your own headline.
%\preprinttext{To appear in IEEE Transactions on Visualization and Computer Graphics.}

%% If you are submitting a paper to a conference for review with a double
%% blind reviewing process, please replace the value ``0'' below with your
%% OnlineID. Otherwise, you may safely leave it at ``0''.
\onlineid{0}

%% declare the category of your paper, only shown in review mode
\vgtccategory{Research}
%% please declare the paper type of your paper to help reviewers, only shown in review mode
%% choices:
%% * algorithm/technique
%% * application/design study
%% * evaluation
%% * system
%% * theory/model
\vgtcpapertype{please specify}

%% Paper title.
\title{Assignment 2 - CS 458}

%% This is how authors are specified in the journal style

%% indicate IEEE Member or Student Member in form indicated below
\author{Amber Horvath, Alannah Oleson, and Katherine Bajno}
\authorfooter{
%% insert punctuation at end of each item
\item
 Amber Horvath, computer science student at Oregon State University: E-mail: horvatha@oregonstate.edu.
\item
 Alannah Oleson, computer science student at Oregon State University. E-mail: olesona@oregonstate.edu.
\item
 Katherine Bajno, computer science student at Oregon State University. E-mail: kbajno@oregonstate.edu
}

%other entries to be set up for journal
%\shortauthortitle{Firstauthor \MakeLowercase{\textit{et al.}}: Paper Title}

%% Abstract section.
% end of abstract

%% Keywords that describe your work. Will show as 'Index Terms' in journal
%% please capitalize first letter and insert punctuation after last keyword
%\keywords{Radiosity, global illumination, constant time}

%% ACM Computing Classification System (CCS). 
%% See <http://www.acm.org/class/1998/> for details.
%% The ``\CCScat'' command takes four arguments.



%% Uncomment below to include a teaser figure.


%% Uncomment below to disable the manuscript note
\renewcommand{\manuscriptnotetxt}{}

%% Copyright space is enabled by default as required by guidelines.
%% It is disabled by the 'review' option or via the following command:
% \nocopyrightspace

\vgtcinsertpkg

%%%%%%%%%%%%%%%%%%%%%%%%%%%%%%%%%%%%%%%%%%%%%%%%%%%%%%%%%%%%%%%%
%%%%%%%%%%%%%%%%%%%%%% START OF THE PAPER %%%%%%%%%%%%%%%%%%%%%%
%%%%%%%%%%%%%%%%%%%%%%%%%%%%%%%%%%%%%%%%%%%%%%%%%%%%%%%%%%%%%%%%%

\begin{document}

%% The ``\maketitle'' command must be the first command after the
%% ``\begin{document}'' command. It prepares and prints the title block.

%% the only exception to this rule is the \firstsection command
%\firstsection{Introduction}

\maketitle

\section{Introduction} %for journal use above \firstsection{..} instead

\subsection{Problem}
Within the last decade, numerous works have surfaced that suggest climate change has detrimental effects on many 
aspects of the environment. One indicator of climate change in a geographic region is air quality, which is measured 
in parts per million of particulate matter. Generally, any particulate less than 2.5 microns in diameter meets the 
standards for “dangerous” particulates. An area that has a high concentration of dangerous particulates - a high 
number on the air quality index, which corresponds to bad air quality - can be both a symptom of or a catalyst for 
climate change. Identifying areas in which the air quality is markedly bad or decreasing over time can provide a way 
to focus climate change studies and environmental science efforts.

\subsection{Motivation}
Though there is much data available on the topic of air quality, very little of it is not simply presented in a table 
or list. Of the visualizations that do exist, many are simply colored maps that make darker or more saturated colors 
correspond to worse air qualities. Thus, it can be hard to see just which areas present a problem over time (
signified by either a continual or sudden, severe decrease in air quality). A visualization that could clearly show 
both the magnitude of the air quality index and give an indication of how the quality was increasing or decreasing 
over time would be very useful to researchers in the field.

\subsection{Potential Users}
Our potential users include researchers and climate change/air quality scientists who study geographic regions in the 
western US. (We will focus our visualization on this region in order to make the scope appropriate for this project.) 
This visualization will ideally give scientists a quick overview of air quality trends in an area, which could 
indicate that the region requires more study or analysis in future work.


In addition, we should not forget that the general population might benefit from a good visualization of this data as 
well. For instance, perhaps a person who has asthma might view the data as part of a decision on whether or not to 
relocate to a certain state. A citizen activist might also be interested in this data in order to raise awareness in 
their region about the dangers of poor air quality. Multiple cases such as these exist and might be well served by 
this visualization.

\subsection{General Approach}
We will pull data from the American Health Rankings site by the United Health Foundation. We will use this to 
aggregate data from the air quality measures of 13 western region states over the past 10 years. We then intend to 
make a 
series of tree maps that visualize two dimensions of the data: the magnitude of the air pollution levels (represented 
by the size of the block; larger = worse) and the rate of change in air quality levels (represented by the color of 
the block; green = decreasing/getting better, red = increasing/getting worse, more saturated = changing faster).

\section{Visualization Tasks}

\begin{itemize}
\item Our visualization aims to address the following questions:
  \begin{itemize}
    \item How fast is the air quality increasing and decreasing in each state?
    \item What states on the west coast are most at risk of bad air quality?
    \item What trends in air quality can we identify in air quality on the west coast over the past 10 years?
    \item Which states can be identified as “danger zones” for further research (air pollution that is quickly increasing)?
  \end{itemize}
\end{itemize}


\section{Related Work}
********** FILL IN WITH PREVIOUS WORK IN THE FIELD OF AIR POLLUTION RESEARCH AND VISUALIZATIONS ***********

\section{Background}
********** FILL IN WITH QUESTIONS WE'RE TRYING TO ANSWER *************

\section{Methods}
\subsection{Data Sources}
**note: should probably just move all these links to citations and just summarize here
**double note: also need to add eastern region states


We will be pulling data from the United Health Foundation’s website that catalogues data about each state by year. We 
will be studying 11 different states in the western US. We chose the western United States as the data set was 
relevant to our team's interest and to keep the project within a manageable scope.
\begin{itemize}
\item Washington: 
http://www.americashealthrankings.org/explore/2015-annual-report/measure/air/state/WA
\item Oregon: http://www.americashealthrankings.org/explore/2015-annual-report/measure/air/state/OR
\item California: http://www.americashealthrankings.org/explore/2015-annual-report/measure/air/state/CA
\item Alaska: http://www.americashealthrankings.org/explore/2015-annual-report/measure/air/state/AK 
\item Hawaii: http://www.americashealthrankings.org/explore/2015-annual-report/measure/air/state/HI
\item Montana: http://www.americashealthrankings.org/explore/2015-annual-report/measure/air/state/MT
\item Wyoming: http://www.americashealthrankings.org/explore/2015-annual-report/measure/air/state/WY
\item Colorado: http://www.americashealthrankings.org/explore/2015-annual-report/measure/air/state/CO
\item New Mexico: http://www.americashealthrankings.org/explore/2015-annual-report/measure/air/state/NM
\item Idaho: http://www.americashealthrankings.org/explore/2015-annual-report/measure/air/state/ID
\item Utah: http://www.americashealthrankings.org/explore/2015-annual-report/measure/air/state/UT
\item Arizona: http://www.americashealthrankings.org/explore/2015-annual-report/measure/air/state/AZ
\item Nevada: http://www.americashealthrankings.org/explore/2015-annual-report/measure/air/state/NV 
\end{itemize}


\subsection{Data Organization}

We set up our data in the form of 3 tables, 1 for the states, one for the years, and one for the air pollution. The 
entity-relationship diagram for these tables can be seen in Figure \ref{fig:ERdiagram}. 
The state name and year are used as keys to index the values stored within the air pollution table. This data was 
retrieved from the United Health Foundation's report on air pollution across the United States.

\begin{figure}
 \centering % avoid the use of \begin{center}...\end{center} and use \centering instead (more compact)
 \includegraphics[width=\columnwidth]{cs458-ER-diagram-ass1}
 \caption{An ER diagram showing the relationship between our 3 tables: the State table has a primary key "name" and the Year table has a primary key "year", both of which are used to query on the Air Pollution table}
 \label{fig:ERdiagram}
\end{figure}


\begin{figure}
\centering
\includegraphics[width=\columnwidth]{state_db}
\caption{A visualization of the contents of the "state" table, with the "name" column serving as the primary key and key to index the "air pollution" table.}
\label{fig:stateDB}
\end{figure}

\begin{figure}
\centering
\includegraphics[width=\columnwidth]{air_poll_db}
\caption{The "air pollution" table's columns include "air pollution", "name", and "year" with "air pollution" 
containing all the values of the air pollution for each state and year, and "name" and "year" serving as keys.}
\label{fig:airPoll}
\end{figure}

\subsection{Design of the Interface}

We represent our data using an interactive tree map on a webpage. On the left hand side, users can select what year 
they want to view from 2007 to 2016, giving them the ability to see the changes in air pollution over the years. The 
size of the boxes on the tree map represent how much or how little air pollution there is in a state. The larger the 
box, the more air pollution exists. The color of the boxes represent the change in air pollution from the previous 
year to the one selected. The range is from -1 to +1, with -1 representing the highest positive change and the +1 
representing the highest negative change in air pollution. The visualization is represented in Figure \ref{fig:Design}
.

\begin{figure}
\centering
\includegraphics[width=\columnwidth]{HW1Design}
\caption{A visualization using a tree map describing the levels of air pollution in the western region by block size.}
\label{fig:Design}
\end{figure}

\subsection{Enhancement Over Existing Models}

** Fill in with why our design is better/more helpful than other already existing models - should tie back to Related Works **
** Idea - whoever writes this section should also be in charge of related work **

\section{Implementation}

\subsection{Data Organization}
We created a relational database with 3 tables to store our data. The "state" table and "year" database are used to
index values stored within the "air pollution" database, as seen in the ER diagram in Figure \ref{fig:ERdiagram}.
We implemented the database on the ONID Database Server provided by Oregon State University. The contents
of the "state" diagram can be seen in Figure \ref{fig:stateDB} and the "air pollution" configuration can be found in Figure \ref{fig:airPoll}.

\subsection{Website}
** fill in with how we created website - might need Kat's help here **

\subsection{Issues Encountered}
** fill in with any issues encountered probably while creating visualization/website - again might need Kat's help here **

\section{Results}
** intro to what we found **

\subsection{Insights}

** ??? I don't really know what kinds of insights we'd find but we will see I guess **

\subsection{Data Set}

** Yeah we already said this up in Methods so we could just reference that again I guess??? **

\subsection{Dimensions Used}

** Our visualization is both spatial (size of block) and temporal (can show change over time) so I guess talk about that/why we chose that visualization/if it worked or not **

\subsection{Performance}

** no idea here since we're not running any sort of tests on it or doing any user studies lmao **

\subsection{Supplementary Materials}
** alannah said: link to source code - so I guess here we could link to our github or something **

\section{Conclusion and Future Work}

** some longform paragraph here about our model, why it's awesome, etc. how we would want to change it in the future **

\section{Acknowledgements}

** TO AMBER FROM AMBER: i think the original tex doc he supplied had some special thing for acknowledgements so copy paste**
** Here we just acknowledge the website we got data from (?) and Eugene and the TAs???? **


\end{document}

